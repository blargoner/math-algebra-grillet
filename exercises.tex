% Notes and exercises from Abstract Algebra, 2nd ed. by Grillet
% By John Peloquin
\documentclass[letterpaper,12pt]{article}
\usepackage{amsmath,amssymb,amsthm,enumitem,diagrams,fourier}

\newcommand{\subgroupeq}{\le}
\newcommand{\nsubgroupeq}{\trianglelefteq}
\newcommand{\lsemidirect}{\rtimes}
\newcommand{\divides}{|}

\newcommand{\after}{\circ}

\newcommand{\ac}{\overline}

\DeclareMathOperator{\Irr}{Irr}

% Theorems
\theoremstyle{definition}
\newtheorem*{defn}{Definition}
\newtheorem*{exer}{Exercise}

\theoremstyle{remark}
\newtheorem*{rmk}{Remark}

\theoremstyle{plain}
\newtheorem*{prop}{Proposition}
\newtheorem*{cor}{Corollary}

% Meta
\title{Notes and exercises from \textit{Abstract Algebra}}
\author{John Peloquin}
\date{}

\begin{document}
\maketitle

\section*{Introduction}
This document contains notes and exercises from~\cite{grillet}.

\section*{Chapter~I}
\subsection*{Section~4}
In addition to Propositions 4.9~and~4.10, the following is useful (see for example the proof of Theorem~II.9.12):
\begin{prop}
Let \(G\)~be a group, \(N\nsubgroupeq G\) and \(N\subseteq H,K\subgroupeq G\). Then \(H\)~and~\(K\) are conjugate in~\(G\) if and only if \(H/N\)~and~\(K/N\) are conjugate in~\(G/N\).
\end{prop}

\section*{Chapter~II}
\subsection*{Section~5}

The argument used in the proof of Proposition~5.10 is essentially Frattini's:
\begin{prop}[Frattini]
Let \(G\)~be a finite group, \(H\nsubgroupeq G\), and \(P\)~a Sylow \(p\)-subgroup of~\(H\). Then \(G=HN_G(P)\).
\end{prop}
\begin{proof}
If \(g\in G\), then \(gPg^{-1}\subseteq gHg^{-1}=H\) since \(P\subseteq H\nsubgroupeq G\). But \(gPg^{-1}\)~is also a Sylow \(p\)-subgroup of~\(H\), and all Sylow \(p\)-subgroups of~\(H\) are conjugate in~\(H\) (Theorem~5.7), so there is \(h\in H\) with
\[hgP(hg)^{-1}=h(gPg^{-1})h^{-1}=P\]
Therefore \(hg\in N_G(P)\), \(g\in HN_G(P)\), and \(G=HN_G(P)\).
\end{proof}
\noindent The key observation is that since all conjugates of~\(P\) in~\(G\) are contained in~\(H\), they are also conjugate in~\(H\). Proposition~5.10 follows as a corollary:
\begin{cor}
Let \(G\)~be a finite group, \(P\)~a Sylow \(p\)-subgroup of~\(G\), and \(N_G(P)\subseteq H\subgroupeq G\). Then \(N_G(H)=H\).
\end{cor}
\begin{proof}
Note \(N_G(H)\)~is finite, \(H\nsubgroupeq N_G(H)\), and \(P\)~is a Sylow \(p\)-subgroup of~\(H\), so \(N_G(H)=HN_{N_G(H)}(P)\subseteq HN_G(P)\subseteq HH=H\) by Frattini.
\end{proof}

\subsection*{Section~9}
\begin{rmk}
In the proof of Lemma~9.11, \(G=N\lsemidirect A=N\lsemidirect B\) (Proposition~11.2). In particular, each \(b\in B\) can be expressed uniquely in~\(N\lsemidirect A\) in the form \(b=ua\) with \(u\in N\) and \(a\in A\). Then \(u=u_a\) in Grillet's notation, and \(u_{aa'}=u_a(au_{a'}a^{-1})\) follows from the multiplication rule in~\(N\lsemidirect A\). In this way, \(N\)~acts as a ``bridge'' between \(A\)~and~\(B\).
\end{rmk}

\subsection*{Section~10}
Commutator subgroups satisfy the following universal mapping property:
\begin{prop}
Let \(G\)~be a group, \(H\nsubgroupeq G\), and \(K=[G,H]\)~the subgroup of~\(G\) generated by commutator elements \([x,y]=xyx^{-1}y^{-1}\) with \(x\in G\) and \(y\in H\). Then \(K\nsubgroupeq G\). If \(\pi:G\to G/K\) is the canonical projection, then \(\pi(H)\subseteq Z(\pi(G))\), and if \(\varphi:G\to L\) is a homomorphism with \(\varphi(H)\subseteq Z(\varphi(G))\), then \(\varphi\)~factors uniquely through~\(\pi\); that is, there exists \(\psi:G/K\to L\) unique such that \(\varphi=\psi\after\pi\):
\begin{diagram}[nohug]
G&\rTo^{\pi}&G/K\\
&\rdTo<{\varphi}&\dTo>{\psi}\\
&&L
\end{diagram}
\end{prop}
\begin{proof}
By the universal mapping property for quotient groups (Theorem~I.5.1), since \(K\subseteq\ker\varphi\).
\end{proof}
\noindent This is a generalization of the universal mapping property noted in Section~9, where \(H=G\) (see Proposition~9.1 and Exercise~9.7). It is implicit in the proofs of Propositions 10.1~and~10.3.

\section*{Chapter~IV}
\subsection*{Section~5}
\begin{rmk}
In Proposition~5.1(2), if \(K\)~is finite then \(m=0\) and \(q\)~is separable.
\end{rmk}

\subsection*{Section~6}
We sketch an alternative approach to purely inseparable extensions starting with polynomials having only one distinct root:
\begin{defn}
A nonconstant polynomial \(f(X)\in K[X]\) is \emph{purely inseparable} if
\[f(X)=a(X-\alpha)^m\in\ac{K}[X]\]
where \(a\in K\), \(\alpha\in\ac{K}\), and \(m>0\).
\end{defn}
\noindent Note \(f\)~is both separable and purely inseparable if and only if \(f\)~is linear.

\begin{prop}
Let \(f(X)=a(X-\alpha)^m\in K[X]\) be purely inseparable as above.
\begin{enumerate}
\item If \(K\)~has characteristic~\(0\), then \(\alpha\in K\).
\item If \(K\)~has characteristic \(p\ne 0\), then \(\alpha^{p^k}\in K\) for some \(k\ge 0\) with
\[f(X)=a\bigl(X^{p^k}-\alpha^{p^k}\bigr)^{m/p^k}\]
\end{enumerate}
\end{prop}
\begin{proof}
By the binomial theorem,
\[f(X)=a(X-\alpha)^m=aX^m-am\alpha X^{m-1}+\cdots\ \in K[X]\]
so \(am\alpha\in K\) and \(m\alpha\in K\) since \(a\ne 0\). If \(K\)~has characteristic~\(0\), then \(m\ne 0\) in~\(K\) and \(\alpha\in K\). If \(K\)~has characteristic \(p\ne 0\), then either \(\alpha\in K\) or else \(p\divides m\) and
\[f(X)=a\bigl((X-\alpha)^p\bigr)^{m/p}=a\bigl(X^p-\alpha^p\bigr)^{m/p}\]
Repeating this argument with \(\alpha^p\)~in place of~\(\alpha\), we must eventually find \(k\ge 0\) with \(\alpha^{p^k}\in K\) and \(f(X)\)~as claimed.
\end{proof}

\begin{prop}
Let \(q(X)\in K[X]\) be monic irreducible and purely inseparable. If \(K\)~has characteristic~\(0\), then \(q(X)=X-a\) for some \(a\in K\). If \(K\)~has characteristic \(p\ne 0\), then \(q(X)=X^{p^k}-a\) for some \(a\in K\) and \(k\ge 0\).
\end{prop}
\begin{proof}
By Proposition~5.1 and the above. In the case of characteristic \(p\ne 0\), \(q(X)=s(X^{p^k})\) for \(s\)~separable and purely inseparable, hence linear.
\end{proof}

\begin{defn}
An element~\(\alpha\) is \emph{purely inseparable over~\(K\)} when \(\alpha\)~is algebraic over~\(K\) and \(\Irr(\alpha:K)\)~is purely inseparable.
\end{defn}

\begin{defn}
An algebraic extension~\(E\) of~\(K\) is \emph{purely inseparable over~\(K\)} when every element of~\(E\) is purely inseparable over~\(K\).
\end{defn}

\noindent These definitions are compatible with those in the text. In particular:

\begin{cor}
An extension~\(E\) of~\(K\) is both separable and purely inseparable over~\(K\) if and only if \(E=K\). In particular if \(K\)~has characteristic~\(0\) or \(K\)~is finite, then \(K\)~is the only purely inseparable extension of~\(K\).
\end{cor}

\begin{cor}
If \(K\)~has characteristic \(p\ne 0\) and \(E\)~is a purely inseparable extension of~\(K\) in~\(\ac{K}\), then
\[E\subseteq K^{1/p^{\infty}}=\bigl\{\,\alpha\in\ac{K}\mid\alpha^{p^k}\in K\text{ for some }k\ge 0\,\bigr\}\]
\end{cor}

% References
\begin{thebibliography}{0}
\bibitem{grillet} Grillet, Pierre~A. \textit{Abstract Algebra}, 2nd~ed. Springer, 2007.
\end{thebibliography}
\end{document}
