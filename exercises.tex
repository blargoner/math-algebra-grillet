% Notes and exercises from Abstract Algebra, 2nd ed. by Grillet
% By John Peloquin
\documentclass[letterpaper,12pt]{article}
\usepackage{amsmath,amssymb,amsthm,enumitem,diagrams,fourier}

\newcommand{\subgroupeq}{\le}
\newcommand{\nsubgroupeq}{\trianglelefteq}
\newcommand{\lsemidirect}{\rtimes}

% Theorems
\theoremstyle{definition}
\newtheorem*{exer}{Exercise}

\theoremstyle{remark}
\newtheorem*{rmk}{Remark}

\theoremstyle{plain}
\newtheorem*{prop}{Proposition}
\newtheorem*{cor}{Corollary}

% Meta
\title{Notes and exercises from \textit{Abstract Algebra}}
\author{John Peloquin}
\date{}

\begin{document}
\maketitle

\section*{Introduction}
This document contains notes and exercises from~\cite{grillet}.

\section*{Chapter~I}
\subsection*{Section~4}
In addition to Propositions 4.9~and~4.10, the following is useful (see for example the proof of Theorem~II.9.12):
\begin{prop}
Let \(G\)~be a group, \(N\nsubgroupeq G\) and \(N\subseteq H,K\subgroupeq G\). Then \(H\)~and~\(K\) are conjugate in~\(G\) if and only if \(H/N\)~and~\(K/N\) are conjugate in~\(G/N\).
\end{prop}

\section*{Chapter~II}
\subsection*{Section~5}

The argument used in the proof of Proposition~5.10 is essentially Frattini's:
\begin{prop}[Frattini]
Let \(G\)~be a finite group, \(H\nsubgroupeq G\), and \(P\)~a Sylow \(p\)-subgroup of~\(H\). Then \(G=HN_G(P)\).
\end{prop}
\begin{proof}
If \(g\in G\), then \(gPg^{-1}\subseteq gHg^{-1}=H\) since \(P\subseteq H\nsubgroupeq G\). But \(gPg^{-1}\)~is also a Sylow \(p\)-subgroup of~\(H\), and all Sylow \(p\)-subgroups of~\(H\) are conjugate in~\(H\) (Theorem~5.7), so there is \(h\in H\) with
\[hgP(hg)^{-1}=h(gPg^{-1})h^{-1}=P\]
Therefore \(hg\in N_G(P)\), \(g\in HN_G(P)\), and \(G=HN_G(P)\).
\end{proof}
\noindent The key observation is that since all conjugates of~\(P\) in~\(G\) are contained in~\(H\), they are also conjugate in~\(H\). Proposition~5.10 follows as a corollary:
\begin{cor}
Let \(G\)~be a finite group, \(P\)~a Sylow \(p\)-subgroup of~\(G\), and \(N_G(P)\subseteq H\subgroupeq G\). Then \(N_G(H)=H\).
\end{cor}
\begin{proof}
Note \(N_G(H)\)~is finite, \(H\nsubgroupeq N_G(H)\), and \(P\)~is a Sylow \(p\)-subgroup of~\(H\), so \(N_G(H)=HN_{N_G(H)}(P)\subseteq HN_G(P)\subseteq HH=H\) by Frattini.
\end{proof}

\subsection*{Section~9}
\begin{rmk}
In the proof of Lemma~9.11, \(G=N\lsemidirect A=N\lsemidirect B\) (Proposition~11.2). In particular, each \(b\in B\) can be expressed uniquely in~\(N\lsemidirect A\) in the form \(b=ua\) with \(u\in N\) and \(a\in A\). Then \(u=u_a\) in Grillet's notation, and \(u_{aa'}=u_a(au_{a'}a^{-1})\) follows from the multiplication rule in~\(N\lsemidirect A\). In this way, \(N\)~acts as a ``bridge'' between \(A\)~and~\(B\).
\end{rmk}

% References
\begin{thebibliography}{0}
\bibitem{grillet} Grillet, Pierre~A. \textit{Abstract Algebra}, 2nd~ed. Springer, 2007.
\end{thebibliography}
\end{document}
