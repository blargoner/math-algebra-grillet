% Notes and exercises from Abstract Algebra, 2nd ed. by Grillet
% By John Peloquin
\documentclass[letterpaper,12pt]{article}
\usepackage{amsmath,amssymb,amsthm,enumitem,diagrams,fourier}

\newcommand{\subgroupeq}{\le}
\newcommand{\nsubgroupeq}{\trianglelefteq}

% Theorems
\theoremstyle{definition}
\newtheorem*{exer}{Exercise}

\theoremstyle{remark}
\newtheorem*{rmk}{Remark}

\theoremstyle{plain}
\newtheorem*{prop}{Proposition}
\newtheorem*{cor}{Corollary}

% Meta
\title{Notes and exercises from \textit{Abstract Algebra}}
\author{John Peloquin}
\date{}

\begin{document}
\maketitle

\section*{Introduction}
This document contains notes and exercises from~\cite{grillet}.

\section*{Chapter~II}
\subsection*{Section~5}

The argument used in the proof of Proposition~5.10 is essentially Frattini's:
\begin{prop}[Frattini]
Let \(G\)~be a finite group, \(H\nsubgroupeq G\), and \(P\)~a Sylow \(p\)-subgroup of~\(H\). Then \(G=HN_G(P)\).
\end{prop}
\begin{proof}
If \(g\in G\), then \(gPg^{-1}\subseteq gHg^{-1}=H\) since \(P\subseteq H\nsubgroupeq G\). But \(gPg^{-1}\)~is also a Sylow \(p\)-subgroup of~\(H\), and all Sylow \(p\)-subgroups of~\(H\) are conjugate in~\(H\) (Theorem~5.7), so there is \(h\in H\) with
\[hgP(hg)^{-1}=h(gPg^{-1})h^{-1}=P\]
Therefore \(hg\in N_G(P)\), \(g\in HN_G(P)\), and \(G=HN_G(P)\).
\end{proof}
\noindent The key observation is that since all conjugates of~\(P\) in~\(G\) are contained in~\(H\), they are also conjugate in~\(H\). Proposition~5.10 follows as a corollary:
\begin{cor}
Let \(G\)~be a finite group, \(P\)~a Sylow \(p\)-subgroup of~\(G\), and \(N_G(P)\subseteq H\subgroupeq G\). Then \(N_G(H)=H\).
\end{cor}
\begin{proof}
Note \(N_G(H)\)~is finite, \(H\nsubgroupeq N_G(H)\), and \(P\)~is a Sylow \(p\)-subgroup of~\(H\), so \(N_G(H)=HN_{N_G(H)}(P)\subseteq HN_G(P)\subseteq HH=H\) by Frattini.
\end{proof}

% References
\begin{thebibliography}{0}
\bibitem{grillet} Grillet, Pierre~A. \textit{Abstract Algebra}, 2nd~ed. Springer, 2007.
\end{thebibliography}
\end{document}
